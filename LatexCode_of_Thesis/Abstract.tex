\chapter*{ABSTRACT}
\addcontentsline{toc}{chapter}{\normalfont Abstract}

This research centers on the comparative examination and data analysis of three gravitational wave events: GW170817, GW190521, and GW190814. The study seeks to examine gravitational wave signals collected by observatories such as LIGO and Virgo using modern computer techniques, including Fourier transforms and numerical simulations. The research entails utilizing fast fourier transform techniques to break down the gravitational wave signals into their individual frequencies, enabling a thorough analysis of the fundamental physics mechanisms responsible for generating these waves. In addition, the study investigates several filtering and noise reduction technique to improve the precision of signal detection. The discoveries enhance our comprehension of astrophysical occurrences, such as the collision of black holes and neutron stars, and offer valuable insights into the wider significance of gravitational wave astronomy.

\vspace{0.5cm}
\noindent {\fontsize{10}{12}\selectfont \textit{Keywords:} General Relativity, Gravitational Waves, Fast Fourier Transform, Windowing, Matched Filtering}


\newpage
\begin{nepali}
	\chapter*{शोधसार}
\end{nepali}
\begin{sanskrit}		
	\noindent \small यो अनुसन्धानले तीनवटा गुरुत्वाकर्षण तरंग घटनाहरू: GW170817, GW190521, र GW190814 को तुलनात्मक परीक्षण र डेटा विश्लेषणमा केन्द्रित छ। यस अध्ययनले LIGO र VIRGO जस्ता वेधशालाहरूद्वारा सङ्कलित गुरुत्वाकर्षण तरंग संकेतहरूलाई फुरियर रूपान्तरण र सङ्ख्यात्मक सिमुलेशन जस्ता आधुनिक कम्प्युटर प्रविधिहरू प्रयोग गरेर जाँच गर्न खोज्छ। अनुसन्धानले तीव्र फुरियर रूपान्तरण प्रविधिहरूलाई प्रयोग गरी गुरुत्वाकर्षण तरंग संकेतहरूलाई तिनीहरूको व्यक्तिगत आवृत्तिहरूमा टुक्रा पार्ने कार्य समावेश गर्दछ, जसले यी तरंगहरू उत्पन्न गर्ने मौलिक भौतिकी संयन्त्रहरूको विस्तृत विश्लेषणलाई सक्षम बनाउँछ। साथै, अध्ययनले सङ्केत पहिचानको शुद्धता सुधार गर्नका लागि विभिन्न फिल्टरिङ र आवाज घटाउने प्रविधिहरूको अनुसन्धान गर्दछ। यस अनुसन्धानले ब्ल्याक होल र न्युट्रोन ताराहरूको टकराव जस्ता खगोलीय घटनाहरूको हाम्रो बुझाइलाई बढाउँछ र गुरुत्वाकर्षण तरंग खगोल विज्ञानको व्यापक महत्त्वमा बहुमूल्य जानकारी प्रदान गर्दछ।
\end{sanskrit}

\vspace{0.5cm}
\noindent {\fontsize{10}{12}\selectfont \textit{Keywords:} General Relativity, Gravitational Waves, Fast Fourier Transform, Windowing, Matched Filtering}