\appendix
\renewcommand{\thesection}{\Alph{section}.}

\chapter*{APPENDIX}
\addcontentsline{toc}{chapter}{\textbf{APPENDIX}}
\onehalfspacing


\section{Python Implementation for Strain, ASD and Qtransform plot}
\label{code1}

\begin{lstlisting}[language=Python, caption=Preprocessing Implementation in python]
""
Written By
    Shivaji Chaulagain
""
from gwosc import datasets
from gwpy.timeseries import TimeSeries, TimeSeriesDict
from matplotlib import pyplot as plt
from pycbc.waveform import get_td_waveform, fd_approximants
import numpy as np
from scipy import signal as sc

class GWanalysis:
  def __init__(self, eventname, duration):
    self.eventname = eventname
    self.duration = duration
    self.gps = datasets.event_gps(eventname)
    self.segment = (int(self.gps)-self.duration/2, int(self.gps)+self.duration/2)
    self.ldata = TimeSeries.fetch_open_data("L1", *self.segment, cache=True)
    self.hdata = TimeSeries.fetch_open_data("H1", *self.segment, cache=True)
    self.vdata = TimeSeries.fetch_open_data("V1", *self.segment, cache=True)
  # plot strain data of given duration.
  def straindataplot(self):
    combined = TimeSeriesDict()
    combined['L1'] = self.ldata
    combined['H1'] = self.hdata
    combined['V1'] = self.vdata
    plot = combined.plot(figsize=(12, 10), separate=True, sharex=True)
    ax1, ax2, ax3 = plot.axes
    ax1.set_title(f'LIGO-Hanford-Virgo strain data around {self.eventname} of duration {self.duration}s')
    ax1.set_ylabel('Amplitude [strain]', y=-0.5)
    ax1.lines[0].set_color('blue')
    ax2.lines[0].set_color('green')
    ax3.lines[0].set_color('red')
    ax1.legend()
    ax2.legend()
    ax3.legend()
    ax2.set_ylabel('')
    ax3.set_ylabel('')
    plot.savefig(f'{self.eventname}strain.png')
  # plot the Amplitude Spectral density. More duration(s) better the resolution of frequency component
  def asdplot(self, fftlength=4, method="median"):
    asdl = self.ldata.asd(fftlength=fftlength, method="median")
    asdh = self.hdata.asd(fftlength=fftlength, method="median")
    asdv = self.vdata.asd(fftlength=fftlength, method="median")
    lasd = asdl.plot()
    ax = lasd.gca()
    ax.set_xlim(10, 1400)
    ax.set_ylim(1e-24, 1e-20)
    ax.plot(asdh, label='LIGO-Hanford', color='gwpy:ligo-hanford')
    ax.plot(asdv, label='Virgo', color='gwpy:virgo')

    lline = ax.lines[0]
    lline.set_color('gwpy:ligo-livingston')  
    lline.set_label('LIGO-Livingston')

    ax.set_ylabel(r'Strain noise [$1/\sqrt{\mathrm{Hz}}$]')
    ax.set_title(f'ASD of each detector around {self.eventname}')
    ax.legend(fontsize=7,loc='lower left')
    plt.savefig(f'{self.eventname}asd.png', dpi=300, bbox_inches='tight')
  # q-transform. I have kept duration of 32 second for signal
  def qtransformplot(self, franges, qranges, outsegs=None):
    if outsegs is not None:
        o1, o2 = outsegs
    ldata = TimeSeries.fetch_open_data("L1", int(self.gps)-30, int(self.gps)+2, cache=True)
    hdata = TimeSeries.fetch_open_data("H1", int(self.gps)-30, int(self.gps)+2, cache=True)
    vdata = TimeSeries.fetch_open_data("V1", int(self.gps)-30, int(self.gps)+2, cache=True)
    if outsegs is None:
        lq = ldata.q_transform(frange=franges, qrange=qranges)
        hq = hdata.q_transform(frange=franges, qrange=qranges)
        vq = vdata.q_transform(frange=franges, qrange=qranges)
    else:
        lq = ldata.q_transform(frange=franges, qrange=qranges, outseg=(self.gps-o1, self.gps+o2))
        hq = hdata.q_transform(frange=franges, qrange=qranges, outseg=(self.gps-o1, self.gps+o2))
        vq = vdata.q_transform(frange=franges, qrange=qranges, outseg=(self.gps-o1, self.gps+o2))
    plot, axes = plt.subplots(nrows=3, sharex=True, figsize=(5, 10))
    tax, qax, qax2 = axes
    tax.imshow(lq)
    qax.imshow(hq)
    qax2.imshow(vq)
    tax.set_title(f"Q-transform around {self.eventname}")
    tax.set_ylabel('Frequency [Hz]', y=-0.5)
    qax.set_ylabel('')
    qax2.set_ylabel('')
    qax2.set_xlabel('Time[seconds]')
    tax.set_yscale('log')
    qax.set_yscale('log')
    qax2.set_yscale('log')
    tick_positions = np.arange(-o1, o2, 0.4)
    # tax.set_xticklabels([])
    # qax.set_xticklabels([])
    # qax2.set_xticklabels([])
    tax.set_xticklabels([round(pos,2) for pos in tick_positions])
    qax.set_xticklabels([round(pos,2) for pos in tick_positions])
    qax2.set_xticklabels([round(pos,2) for pos in tick_positions])
    # show the detector name in plot
    tax.text(0.05, 0.95, 'LIGO-Livingston', transform=tax.transAxes, fontsize=7, verticalalignment='top', color='white')
    qax.text(0.05, 0.95, 'LIGO-Hanford', transform=qax.transAxes, fontsize=7, verticalalignment='top', color='white')
    qax2.text(0.05, 0.95, 'Virgo', transform=qax2.transAxes, fontsize=7, verticalalignment='top', color='white')
    # hide gid
    tax.grid(False)
    qax.grid(False)
    qax2.grid(False)
    tax.colorbar(clim = (0,20),label='Normalised energy')
    qax.colorbar(clim = (0,20))
    qax2.colorbar(clim = (0,20))

    plt.savefig(f'{self.eventname}_qtransform.png', dpi=300, bbox_inches='tight')
    
\end{lstlisting}
\newpage
\section{Python Implementation for Waveform Generation (Template)}
\label{code2}

For information about the conditioned variable, what the below code is and how it is obtained and the remaining code for whitening and template generation, see the following references in the gw-odw2023 GitHub repository: \url{https://github.com/gw-odw/odw-2023/blob/main/Tutorials/Day_2/Tuto_2.2_Matched_Filtering_In_action.ipynb}. Furthermore, about my work, see my github repository: \url{https://github.com/Shivaji-137}
\subsection{GW190521}
\begin{lstlisting}
conditioned = strain.crop(2, 2) #obtained from strain data
from pycbc.waveform import get_td_waveform
hp, hc = get_td_waveform(approximant="IMRPhenomPv3HM",
                     mass1=90,
                     mass2= 65,
                     distance = 4600,
                     delta_t=conditioned.delta_t,
                     f_lower=20)

# Resize the vector to match our data
hp.resize(len(conditioned))
\end{lstlisting}
\subsection{GW190814}
\begin{lstlisting}
from pycbc.waveform import get_td_waveform
hp, hc = get_td_waveform(approximant="IMRPhenomPv2_NRTidal",
                     mass1=2.36,
                     mass2= 2.36,
                     distance = 42,
                     delta_t=conditioned.delta_t,
                     f_lower=60)

# Resize the vector to match our data
hp.resize(len(conditioned))
\end{lstlisting}
 




