\chapter{CONCLUSIONS AND RECOMMENDATIONS}
\onehalfspacing

\section{Conclusions}
We have analysed three gravitational wave events namely GW170817, GW190521, and GW1908\\14 employing strain data, Amplitude Spectral Density plots, Q-transform spectrograms, Signal-to-Noise Ratio analysis and template matching. 
 
%  Due to the fact that GW170817 was a binary neutron star merger, there was a strong chirp signal in the time series analysis and this was clearly evident in the LIGO-Livingston detector. This event also demonstrated LIGO detectors’ high sensitivity especially in the lower frequencies. On the other hand, GW190521 that is associated with the merger of two big black holes released a short and low frequency signal consistent with the energetic and short-lived process. Observations of the GW190814 which probably involved a binary black-hole with a higher mass ratio than the other binary black-holes showed a signal that was as long and frequent as that of neutron star and black-hole merger signals.

% The LIGO-Livingston strains recorded for GW170817 were the lowest over a broad frequency band and especially below 100 Hz, which makes the detector more sensitive to neutron star merger signals. On the other hand, both GW190521 and GW190814, which involved higher frequency emissions, brought attention to the wider noise range in the Virgo detector where noticeable tonal peaks could be observed. The sensitivity of L1 LIGO detector was better in all three events while the Virgo detector was found to have higher noise especially in the lower frequencies.

% The spectrogram of GW170817 showed the characteristic “chirp” signal, where the frequency increased as the neutron stars moved closer to merge. This signal, however, was most explicit in the L1 data. However, We found some loud instrumental noise or glitches in L1 detector. On the other hand, GW190521’s spectrogram showed a short single, low frequency centered around 60 Hz, which is suggestive of massive black hole merging. The spectrogram of GW190814 also contains the chirp like signal, but it also demonstrates the dynamics of a high mass ratio binary, which begins at the low frequencies and gradually increases to represent the merger of the binary.

%  Similarly, GW190521 had the largest peak SNR of 8.95 in the L1 detector which shows that the detector has a good efficiency of detection even if the signal is short. Conversely, the maximum SNR for GW190814 was found to be 6.89, although lower, still endorse a good detection. The template matching for both events, with numerical relativity waveforms, revealed a very good match between the experiments and the theories, thus confirming the general theory of relativity. Specifically, the waveform approximants employed – ‘IMRPhenomPv3HM’ for GW190521 and ‘IMRPhenomPv2\_NRTidal’ for GW190814, brought out the need for different models based on the mass and dynamics of merging objects.

%  A comparison of these events reveals the differences in the gravitational wave signals and the ability of the detectors to pick different types of signals. GW170817 is a relatively short binary neutron star merger signal, which is within the LIGO’s detectable frequency range at the lower end. Conversely, the higher mass black hole mergers GW190521 and GW190814 generated shorter and lower frequency signals with the GW190521 signal being difficult to detect due to its short duration and low frequency. The findings highlight the importance of multi-detector detections in covering the entire range of the gravitational waves and stress the ongoing work on the increases of the detectors’ sensitivity and the improvement of the analysis capabilities.

The comparisons revealed distinct differences based on the nature of the sources. For instance, the binary neutron star merger GW170817 produced a strong chirp signal, which was most evident in the LIGO-Livingston detector. This contrasted with GW190521, a merger of massive black holes that resulted in a short, low-frequency signal. GW190814, which likely involved a binary black hole with a higher mass ratio, exhibited a signal longer and more frequent than that of GW190521, but still different from the neutron star merger.

The comparison of strain data showed that LIGO-Livingston had the lowest noise levels, especially for GW170817, which made it more sensitive to lower frequency signals typical of neutron star mergers. On the other hand, Virgo exhibited higher noise levels, particularly noticeable in the higher frequency emissions of GW190521 and GW190814.

In comparing the spectrograms, the characteristic chirp of GW170817 was most clearly visible in the LIGO-Livingston data, despite some instrumental noise. For GW190521, the spectrogram highlighted a brief, low-frequency signal, consistent with the merger of massive black holes, while GW190814’s spectrogram captured a unique chirp-like signal indicative of a high mass ratio binary.

The SNR analysis further emphasized these differences, with GW190521 showing the highest peak SNR of 8.95, indicating robust detection despite the short signal. Although the SNR for GW190814 was lower, it still confirmed a good detection. Template matching with numerical relativity waveforms demonstrated strong correlations, validating Einstein's General Theory of Relativity. The need for different waveform approximants for these events underscored the importance of using appropriate models based on the mass and dynamics of the merging objects.

By comparing these events, We were able to identify how different gravitational wave signals and detector sensitivities affect the detection and analysis of these cosmic events. The results stress the importance of multi-detector detections to cover the full spectrum of gravitational wave signals and highlight ongoing efforts to enhance detector sensitivity and improve analysis techniques.

\section{Limitations}
The study also had some limitations that affected some of the results in the study. First, a high computational cost is another factor that prevented the authors from undertaking more complex data analyses, which would have enabled them to delve deeper into the issue at hand. The analysis was also sensitive to noise, which could lead to large errors in the results. In particular, glitches which were apparent, especially in LIGO-Livingston data during GW170817 brought significant difficulties in the correct signal analysis and interpretation of the strain data and spectrograms. In addition, the waveform templates in this paper were constructed with certain approximants and mass parameters that might not describe the full details of the events to which they were matched, which could affect the template matching. Finally, template matching was not applicable for the GW170817 event due to the problems with the parameters for the template generation and because of this, the comparative analysis for this event was limited.

\section{Recommendations For Future Work}
Future work should include SNR and template matching for the GW170817 event to provide a more complete comparative analysis across all three events, as the current study faced difficulties in adjusting parameters for template generation, limiting the scope of comparison. Additionally, further investigation is needed to explore advanced deep learning methods, such as convolutional and recurrent neural networks, to improve the detection, classification, and estimation of gravitational waves. Integrating these deep learning approaches with conventional techniques in real-time data processing could significantly enhance the speed and precision of observatory operations. Utilizing sophisticated noise reduction techniques, including machine learning algorithms, may also improve the sensitivity of detectors. Furthermore, employing Bayesian inference and machine learning to refine parameter estimates could provide deeper insights into the dynamics of binary systems. 


