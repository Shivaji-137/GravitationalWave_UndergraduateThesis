The exploration of gravitational waves represents a revolutionary chapter in the field of astrophysics. It was in the early 20th century, through Albert Einstein's profound insights into the nature of spacetime, that the concept of these gravitational ripples was first articulated \cite{einstein1916naherungsweise}. According to Einstein's theory of general relativity, massive objects, when in motion, could cause the very fabric of spacetime to ripple, propagating waves that carry with them information about the universe's most energetic and enigmatic events.
\vspace{0.3cm}

Over the past century, scientific instruments and techniques have evolved to the point where we can now detect and analyze these elusive gravitational waves. Among the myriad sources of these waves, binary systems—comprising pairs of neutron stars, black holes, or a combination of both—hold a particularly captivating allure. These systems, locked in gravitational embrace, produce an intricate cosmic dance as they spiral inward, culminating in a cataclysmic merger.
\vspace{0.3cm}

Of particular interest is the coalescence of binary neutron star-binary black hole systems. These events promise a wealth of scientific insights, as they involve the merger of two of the most extreme celestial entities known to science. Neutron stars are incredibly dense remnants of massive stars, while black holes represent regions of spacetime so curved that nothing, not even light, can escape their grasp. The collision and merger of such exotic objects create gravitational waves that can provide a treasure trove of information about the nature of these systems, the properties of matter at extreme densities, and the fundamental laws of physics governing the cosmos.\\
The theoretical underpinnings of gravitational waves, as described by general relativity, provide a solid foundation for understanding these phenomena. However, their detection and interpretation require sophisticated data analysis techniques, which have led to the development of groundbreaking observatories like the Laser Interferometer Gravitational-Wave Observatory (LIGO) and Virgo. The Laser Interferometer Gravitational-Wave Observatory (LIGO), together with the Virgo Collaboration, made history in 2015 by detecting the first gravitational waves, originating from the collision and merger of two black holes \cite{abbott2016observation}. This monumental discovery confirmed Einstein's theory and opened a new era in astrophysics.