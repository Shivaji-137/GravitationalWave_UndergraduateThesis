\section{Literature Review}
A century ago, Einstein proposed gravitational waves as spacetime ripples resulting from mass acceleration, although doubts lingered. Detecting these waves posed a formidable challenge spanning decades. Additionally, Einstein's theory predicted the existence of black holes—cosmic entities formed through intense spacetime curvature. The understanding of black holes became closely linked with the pursuit of gravitational wave detection. It wasn't until the late 1950s that calculations began to solidify the belief in the existence of gravitational waves. Finally, in the 1970s, indirect evidence emerged when astronomers observed a double pulsar system PSR 1913+16 \citep{1975ApJ...195L..51H} for which measurements of the decay of the orbital period with time are consistent with the energy losses expected for gravitational-wave emission, confirming the energy loss predicted by gravitational wave theory.
\vspace{0.2cm}

On September 14, 2015, the LIGO collaboration made history by detecting GW150914, the first gravitational wave event. This monumental discovery, results from the merger of two massive black holes \citep{abbott2016observation}.
\vspace{0.2cm}

Further, On August 17, 2017, the Advanced LIGO and Advanced VIRGO detectors observed GW170817, marking the first binary neutron star inspiral detection. The signal had a high signal-to-noise ratio 32.4 and a rare false-alarm rate. It suggested component masses between 0.86 and 2.26 $M_\odot$ or 1.17 to 1.60 $M_\odot$ when considering neutron star spins. The total system mass was estimated at 2.74 $M_\odot$. The event was localized within 28 square degrees, the closest such localization. It was associated with gamma-ray burst GRB 170817A, linking neutron star mergers to short gamma-ray bursts \citep{2017ApJ...848L..12A}.
\vspace{0.2cm}

In May 2019, the Advanced LIGO and Advanced VIRGO detectors observed the gravitational wave event GW190521 \citep{abbott2020gw190521}\citep{martin2020gw190521}, characterized by a high signal-to-noise ratio and a low estimated false-alarm rate. This event is believed to result from the merger of two black holes, with estimated masses of approximately 85-66 $M_\odot$. The primary black hole's mass is likely within a range produced by pair-instability supernova processes. The remnant black hole has an estimated mass of about 142 $M_\odot$, classifying it as an intermediate mass black hole. The remaining 9 $M_\odot$ were radiated away as energy in the form of gravitational waves \citep{abbott2020properties}\citep{abbott2020gw190521}. The source of GW190521 is located at a luminosity distance of $5.3 Gpc$, corresponding to a redshift of 0.82. The inferred rate of similar black hole mergers in the universe is approximately 0.13 $Gpc^{-3} yr^{-1}$. These findings provide valuable insights into black hole mergers and their astrophysical implications.
\vspace{0.2cm}

GW190814, observed on August 14, 2019, \citep{abbott2020gw190814} by LIGO and VIRGO, represents a significant gravitational wave event. It involves a compact binary coalescence featuring an unequal mass ratio, with a black hole ranging from 22.2 to 24.3 $M_\odot$ and a compact object between 2.50 and 2.67 $M_\odot$. The signal boasted a high signal-to-noise ratio and was localized to a distance of 241 Mpc \citep{abbott2020gw190814}\citep{staff2020mysteryobject}\citep{staff2020gw190814}. This event challenges current astrophysical models due to its unique mass ratio, component masses, and estimated merger rate density, which ranges from 1 to 23 $Gpc^{-3} yr^{-1}$. Furthermore, tests of general relativity revealed no deviations, confirming its predictions for higher-multipole emission. The origin and characteristics of GW190814 raise intriguing questions in the field of compact-object binary formation \cite{abbott2020gw190814}\citep{staff2020mysteryobject}\citep{staff2020gw190814}.
