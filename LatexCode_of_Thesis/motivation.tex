The research project is deeply rooted in the transformative impact of gravitational wave astronomy on our understanding of the universe. Gravitational wave detectors, such as LIGO and Virgo, have ushered in a new era of exploration, enabling us to directly observe and interpret the gravitational whispers of the cosmos' most cataclysmic events. Among these celestial phenomena, binary systems composed of neutron stars and black holes stand as captivating and mysterious cosmic laboratories where the boundaries of our understanding are continually pushed to the extreme. I intend to study these events: GW170817, GW190521,and GW190814.
\vspace{0.4cm}

\textbf{GW170817}:
Studying GW170817 is crucial as it revolutionized astronomy. It was the first event observed in both gravitational waves and light, showcasing the power of using different methods to study space. This event also provided insights into neutron stars and confirmed how heavy elements like gold are formed in space, advancing our understanding of the cosmos.
\vspace{0.4cm}

\textbf{GW190521}:
GW190521, observed as a binary black hole merger, captured attention due to its unique characteristics. It featured an unusual mass ratio and is believed to have involved the formation of an intermediate-mass black hole. This event challenges conventional models of black hole evolution and dynamics, adding to the intrigue of our study.
\vspace{0.4cm}

\textbf{GW190814}:
GW190814 remains an intriguing enigma in the realm of gravitational wave astronomy. This event, which may have involved a neutron star-black hole binary, presents a unique challenge. Our aim is to uncover the true nature of the merger, further our understanding of compact object binaries, and potentially unveil new astrophysical phenomena.
\vspace{0.3cm}

Through these three events, the correlation among the gravitational wave events GW170817, GW190521, and GW190814 can be studied by comparing their waveforms, source properties, and astrophysical implications. Analyzing these events together allows us to identify commonalities and differences in the characteristics of binary systems, such as mass distributions, spins, and orbital dynamics. This comparative analysis can deepen our understanding of compact object mergers and their broader astrophysical context, enhancing our knowledge of extreme cosmic phenomena.
